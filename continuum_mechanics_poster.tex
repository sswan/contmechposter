\documentclass[10pt]{article}
\usepackage{amssymb,amsmath}
\usepackage{undertilde}
\usepackage{array}
\usepackage{fix-cm}  % for huge title
%\usepackage{txfonts}

% Europe paper
%\usepackage[a1paper,margin=1cm]{geometry}

%\usepackage{rmb_mathTypesetting}

% This sets the size of the paper with a 1 inch margin
%
%  US PAPER SIZE
%
\usepackage[margin=1in]{geometry}
\special{papersize=24in,36in}
\setlength{\paperwidth}{24in}
\setlength{\paperheight}{36in}
\setlength{\textwidth}{22in}
\setlength{\textheight}{34in}

% end special formatting for paper size

% mnemonic "Tensor Underline (Underline, Underline...)"
\newcommand{\tu}[1]{\utilde{#1}}
\newcommand{\tuu}[1]{\utilde{\utilde{#1}}}
\newcommand{\tuuu}[1]{\utilde{\utilde{\utilde{#1}}}}
\newcommand{\tuuuu}[1]{\utilde{\utilde{\utilde{\utilde{#1}}}}}

\DeclareMathOperator{\myskew}{skw}
\newcommand{\skw}[1]{\myskew(#1)}
\DeclareMathOperator{\mysym}{sym}
\newcommand{\sym}[1]{\mysym(#1)}

\DeclareMathOperator{\mytrace}{tr}
\newcommand{\tr}[1]{\mytrace(#1)}

\newlength{\mysep}
\setlength{\mysep}{1ex}

\pagestyle{empty}




\begin{document}

\fontsize{13pt}{13pt}\selectfont
%\fontsize{18pt}{24pt}\selectfont
\begin{center}
\begin{tabular}{l m{3in} c c c c m{5in}}
\multicolumn{7}{c}{\fontsize{72pt}{72pt}\selectfont Continuum Mechanics Equation Sheet} \\

%\\[-5ex]
\\[72pt]  % space between title and headers
Name & Purpose & Symbols & Tensor Notation & Indicial Notation & Units & Notes \\
\\[-1ex]
% #24 ----------------------------------------------------------------------
\hline
\\[-1ex]
% Name
Initial Particle Position
&% Purpose
-
&% My Symbol
$ \tu{X} $
&% Equation
$ \tu{X} $
&% Equation
$ X_{i} $
&% Units
$ length $
&% Notes
Given
\\[1ex]

% #24 ----------------------------------------------------------------------
\hline
\\[-1ex]
% Name
Current Particle Position
&% Purpose
-
&% My Symbol
$ \tu{x} $
&% Equation
$ \tu{x} = \tuu{F} \cdot \tu{X} + \tu{c}$
&% Equation
$ x_{i} = F_{ij} X_{j} + c_{i}$
&% Units
$ length $
&% Notes
The vector $\tu{c}$ is the displacement vector
\\[1ex]

% #24 ----------------------------------------------------------------------
\hline
\\[-1ex]
% Name
Deformation Gradient
&% Purpose
Describes local volume, orientation, and shape changes
&% My Symbol
$ \tuu{F} $
&% Equation
$ \tuu{F}=\frac{\partial \tu{x}}{\partial \tu{X}} = \tu{x}\overleftarrow{\nabla}_{0} = \tuu{R} \cdot \tuu{U} = \tuu{V} \cdot \tuu{R}$
&% Equation
$ F_{ij} =\frac{\partial x_{i}}{\partial X_{j}} = R_{ik} U_{kj} = V_{ik} R_{kj}$
&% Units
$ 1 $
&% Notes
Invertible; Work conjugate with $\tuu{P} \rightarrow \frac{1}{\rho_0}\tuu{P}\colon\tuu{\dot{F}}$
\\[1ex]

% #24 ----------------------------------------------------------------------
\hline
\\[-1ex]
% Name
Jacobian
&% Purpose
Describes local volume changes
&% My Symbol
$ J $
&% Equation
$ J = \det(\tuu{F}) = \det(\tuu{U}) = \det(\tuu{V}) = \frac{V}{V_{0}} = \frac{\rho_{0}}{\rho}$
&% Equation
-
&% Units
$ 1 $
&% Notes
Always positive nonzero; `Jacobian of the motion'
\\[1ex]




% #07 ----------------------------------------------------------------------
%
\hline
\\[-1ex]
% Name
Rotation Tensor
&% Purpose
Describes local rotation
&% My Symbol
$ \tuu{R} $
&% Equation
$ \tuu{R} = \tuu{F} \cdot \tuu{U}^{-1} $
&% Equation
$ R_{ij} = F_{ik} U^{-1}_{kj} $
&% Units
$ 1 $
&% Notes
Orthogonal; Unique; Pure rotation ($\det(\tuu{R}) = 1$); `Polar rotation tensor'
\\[1ex]




% #07 ----------------------------------------------------------------------
%
\hline
\\[-1ex]
% Name
Left stretch
&% Purpose
Spatial stretch measure
&% My Symbol
$ \tuu{V} $
&% Equation
$ \tuu{V} = \tuu{F} \cdot \tuu{R}^{T} $
&% Equation
$ V_{ij} = F_{ik} R_{jk} $
&% Units
$ 1 $
&% Notes
Symmetric; Positive Definite; Unique
\\[1ex]



% #07 ----------------------------------------------------------------------
%
\hline
\\[-1ex]
% Name
Right stretch
&% Purpose
Reference stretch measure
&% My Symbol
$ \tuu{U} $
&% Equation
$ \tuu{U} = \tuu{R}^{T} \cdot \tuu{F} $
&% Equation
$ U_{ij} = R_{ki} F_{kj} $
&% Units
$ 1 $
&% Notes
Symmetric; Positive Definite; Unique; Unaffected by superimposed rotation; `Reference stretch'
\\[1ex]

% #07 ----------------------------------------------------------------------
%
\hline
\\[-1ex]
% Name
Left Cauchy-Green Tensor
&% Purpose
Reference stretch measure
&% My Symbol
$ \tuu{B} $
&% Equation
$ \tuu{B} = \tuu{F} \cdot \tuu{F}^{T} = \tuu{V}^{2}$
&% Equation
$ B_{ij} = F_{ik} F_{jk} = V_{ik} V_{kj}$
&% Units
$ 1 $
&% Notes
Symmetric; Positive Definite; `Finger Tensor'; Inverse called `Cauchy deformation tensor' $\tilde{\tuu{B}}$; `Spatial stretch'
\\[1ex]




% #08 ----------------------------------------------------------------------
\hline
\\[-1ex]
% Name
Right Cauchy-Green Tensor
&% Purpose
Spatial stretch measure
&% My Symbol
$ \tuu{C} $
&% Equation
$ \tuu{C} = \tuu{F}^{T} \cdot \tuu{F} = \tuu{U}^{2}$
&% Equation
$ C_{ij} = F_{ki} F_{kj} = U_{ik} U_{kj}$
&% Units
$ 1 $
&% Notes
Symmetric; Positive Definite; Unaffected by superimposed rotation
\\[1ex]


% #09 ----------------------------------------------------------------------
\hline
\\[-1ex]
% Name
Euler Strain
&% Purpose
Measure of spatial strain
&% My Symbol
$ \tuu{e} $
&% Equation
$ \tuu{e} = \frac{1}{2}(\tuu{I}-\tuu{B}^{-1})=\frac{1}{2}(\tuu{I}-\tuu{F}^{-T}\cdot\tuu{F}^{-1}) $
&% Equation
$ e_{ij} = \frac{1}{2}(\delta_{ij} - B^{-1}_{ij}) = \frac{1}{2}(\delta_{ij} - F^{-1}_{ki} F^{-1}_{kj}) $
&% Units
$ 1 $
&% Notes
Symmetric; Seth-Hill parameter $\kappa = -2$; `Alamansi-Hamel strain tensor'; `Eulerian strain tensor'
\\[1ex]

% #09 ----------------------------------------------------------------------
\hline
\\[-1ex]
% Name
Logarithmic Strain
&% Purpose
Measure of reference strain
&% My Symbol
$ \tuu{\varepsilon} $
&% Equation
$ \tuu{\varepsilon} = \ln(\tuu{U}) $
&% Equation
-
&% Units
$ 1 $
&% Notes
Symmetric; Seth-Hill parameter $\kappa \rightarrow 0$; `Hencky strain tensor'
\\[1ex]





% #10 ----------------------------------------------------------------------
\hline
\\[-1ex]
% Name
Lagrange Strain
&% Purpose
Measure of reference strain
&% My Symbol
$ \tuu{E} $, $ \tuu{\varepsilon}$
&% Equation
$ \tuu{E} = \frac{1}{2}(\tuu{C}-\tuu{I}) = \frac{1}{2}(\tuu{F}^{T}\cdot\tuu{F}-\tuu{I}) $
&% Equation
$ E_{ij} = \frac{1}{2}(C_{ij} - I_{ij}) = \frac{1}{2}(F_{ki} F_{kj} - I_{ij}) $
&% Units
$ 1 $
&% Notes
Symmetric; Seth-Hill parameter $\kappa = 2$; Unaffected by superimposed rotation; Work conjugate with $\tuu{S} \rightarrow \frac{1}{\rho_0}\tuu{S}\colon\dot{\tuu{E}}$; `Green strain tensor'; `Green-St. Venant strain tensor'
\\[1ex]

% #13 ----------------------------------------------------------------------
\hline
\\[-1ex]
% Name
Velocity Gradient
&% Purpose
-
&% My Symbol
$ \tuu{L} $
&% Equation
$ \tuu{L} = \tu{v}\overleftarrow{\nabla}_{x} = \left(\overrightarrow{\nabla}_{x}\tu{v}\right)^{T} = \frac{\partial \tu{v}}{\partial \tu{x}}= \tuu{\dot{F}}\cdot\tuu{F}^{-1} $
&% Equation
$ L_{ij} = \frac{\partial v_{i}}{\partial x_{j}} = \dot{F}_{ik} F^{-1}_{kj} $
&% Units
$ \frac{1}{Second} $
&% Notes
`Spatial velocity gradient'
\\[1ex]


% #12 ----------------------------------------------------------------------
\hline
\\[-1ex]
% Name
Symmetric Part of the Velocity Gradient
&% Purpose
Strain rate approximation
&% My Symbol
$ \utilde{\utilde{D}} $
&% Equation
$ \utilde{\utilde{D}} = \frac{1}{2}(\utilde{\utilde{L}}+\utilde{\utilde{L}}^{T}) = \sym{\utilde{\utilde{L}}} $
&% Equation
$ D_{ij} = \frac{1}{2}(L_{ij} + L_{ji}) = \frac{1}{2}(\delta_{ik}\delta_{jl} + \delta_{il}\delta_{jk}) L_{kl}$
&% Units
$ 1 $
&% Notes
Generally not a true rate; If $\tuu{U}(t)$ is diagonal, $\tuu{D} = \tuu{\dot{\varepsilon}}^{log}$; Work conjugate with $\tuu{\sigma} \rightarrow \frac{1}{\rho}\tuu{\sigma}\colon\tuu{D}$; `Deformation rate'
\\[1ex]

% #11 ----------------------------------------------------------------------
\hline
\\[-1ex]
% Name
Vorticity Tensor
&% Purpose
Measure of `tumble'
&% My Symbol
$ \tuu{W} $
&% Equation
$ \tuu{W} = \frac{1}{2}(\tuu{L} - \tuu{L}^{T}) = \skw{\tuu{L}} $
&% Equation
$ W_{ij} = \frac{1}{2}(L_{ij} - L_{ji}) = \frac{1}{2}(\delta_{ik}\delta_{jl} - \delta_{il}\delta_{jk})\tuu{L}_{kl}$
&% Units
$ \frac{1}{time} $
&% Notes
Skew; `Spin Tensor'
\\[1ex]


% #14 ----------------------------------------------------------------------
\hline
\\[-1ex]
% Name
Vorticity Vector
&% Purpose
Measure of `tumble'
&% My Symbol
$ \tu{w} $
&% Equation
$ \tu{w} = -\frac{1}{2}\tuuu{\varepsilon}\colon\tuu{W} = \frac{1}{2}\overrightarrow{\nabla} \times \tu{v}$
&% Equation
$ w_{i} = -\frac{1}{2} \varepsilon_{ijk} W_{jk} = \frac{1}{2} \varepsilon_{ijk} \frac{\partial v_{k}}{\partial x_{j}}$
&% Units
$ 1 $
&% Notes
Axial vector of $\tuu{W} \rightarrow W_{ij} = -\varepsilon_{ijk} w_{k}$
\\[1ex]


% #04 ----------------------------------------------------------------------
\hline
\\[-1ex]
% Name
Cauchy Stress
&% Purpose
Current force per unit deformed area
&% My Symbol
$ \tuu{\sigma} $
&% Equation
-
&% Equation
-
&% Units
$ \frac{Force}{Area} $
&% Notes
Symmetric; Work conjugate with $\tuu{D} \rightarrow \frac{1}{\rho}\tuu{\sigma}\colon\tuu{D}$; Defined by $\tu{t} = \tuu{\sigma}\cdot\tu{n}$
\\[1ex]


% #04 ----------------------------------------------------------------------
\hline
\\[-1ex]
% Name
First Piola-Kirchhoff Stress
&% Purpose
Current force per unit undeformed area
&% My Symbol
$ \tuu{P} $, $\tuu{t}$
&% Equation
$ \tuu{P} = \tuu{\sigma}\cdot\tuu{F}^{c}  = J \tuu{\sigma}\cdot\tuu{F}^{-T}$
&% Equation
$ P_{ij} = \sigma_{ik} F^{c}_{kj}  = J \sigma_{ik} F^{-1}_{jk}$
&% Units
$ \frac{Force}{Area} $
&% Notes
`Nominal stress tensor'; `Lagrangian stress tensor'; Work conjugate with $\tuu{F} \rightarrow \frac{1}{\rho_0}\tuu{P}\colon\dot{\tuu{F}}$; Solve using Nanson's Relation ($d\tu{A}=\tuu{F}^{c}\cdot d\tu{A}_{0})$)
\\[1ex]



% #05 ----------------------------------------------------------------------
\hline
\\[-1ex]
% Name
Second Piola-Kirchhoff Stress
&% Purpose
Transformed (unrotated) current force per unit undeformed area
&% My Symbol
$ \tuu{S} $
&% Equation
$ \tuu{S} = \tuu{F}^{-1}\cdot\tuu{P} = J \tuu{F}^{-1} \cdot \tuu{\sigma} \cdot \tuu{F}^{-T}$
&% Equation
$ S_{ij} = F^{-1}_{ik} P_{kj} = J F^{-1}_{ik} \sigma_{kl} F^{-T}_{lj}$
&% Units
$ \frac{Force}{Area} $
&% Notes
Symmetric; Work conjugate with $\tuu{E} \rightarrow \frac{1}{\rho_0}\tuu{S}\colon\dot{\tuu{E}}$; Unaffected by superimposed rotation
\\[1ex]



% #15 ----------------------------------------------------------------------
\hline
\\[-1ex]
% Name
Stiffness
&% Purpose
Constitutive relation
&% My Symbol
$ \tuuuu{C}, \tuuuu{\mathbb{E}} $
&% Equation
$ \tuuuu{C} = \frac{\partial \tuu{\sigma}}{\partial \tuu{\varepsilon}} $
&% Equation
$ C_{ijkl} = \frac{\partial \sigma_{ij}}{\partial \varepsilon_{kl}} $
&% Units
$ \frac{Force}{Area} $
&% Notes
For symmetric $\tuu{\sigma}$ and $\tuu{\varepsilon}$, $\tuuuu{C}$ is minor symmetric; `Elastic tangent stiffness'
\\[1ex]
% #16 ----------------------------------------------------------------------
\hline
\\[-1ex]
% Name
Compliance
&% Purpose
Constitutive relation
&% My Symbol
$ \tuuuu{S}, \tuuuu{H}$
&% Equation
$ \tuuuu{S} = \frac{\partial \tuu{\varepsilon}}{\partial \tuu{\sigma}} $
&% Equation
$ S_{ijkl} = \frac{\partial \varepsilon_{ij}}{\partial \sigma_{kl}} $
&% Units
$ \frac{Area}{Force} $
&% Notes
For symmetric $\tuu{\sigma}$ and $\tuu{\varepsilon}$, $\tuuuu{S}$ is minor symmetric
\\[1ex]
% #17 ----------------------------------------------------------------------
\hline
\\[-1ex]
% Name
Specific Kinetic Energy
&% Purpose
-
&% My Symbol
$ k $
&% Equation
$ k =\frac{1}{2}\utilde{v}\cdot\utilde{v}$
&% Equation
$ k =\frac{1}{2}\utilde{v}\cdot\utilde{v}$
&% Units
$ \frac{Length^{2}}{Second^{2}} $
%&% Derivation
%-
&% Notes
$ \utilde{v}\cdot\utilde{a}=\utilde{v}\cdot\dot{\utilde{v}}=\dot{k} $
\\[1ex]
% #18 ----------------------------------------------------------------------
\hline
\\[-1ex]
% Name
Isotropic Stress
&% Purpose
Measure of `average' stress
&% My Symbol
$ iso\utilde{\utilde{A}} $
&% Equation
$ iso\utilde{\utilde{A}} = \frac{1}{3}tr(\utilde{\utilde{A}}) $
&% Equation
$ iso\utilde{\utilde{A}} = \frac{1}{3}tr(\utilde{\utilde{A}}) $
&% Units
$ - $
%&% Derivation
%$ \utilde{\utilde{A}} = iso\utilde{\utilde{A}} + dev\utilde{\utilde{A}} $
&% Notes
A.K.A. `Spherical Stress', `Hydrostatic Stress'
\\[1ex]
% #19 ----------------------------------------------------------------------
\hline
\\[-1ex]
% Name
Deviatoric Stress
&% Purpose
Measure of `shear' stress
&% My Symbol
$ dev\utilde{\utilde{A}} $
&% Equation
$ dev\utilde{\utilde{A}} =  \utilde{\utilde{A}}-iso\utilde{\utilde{A}} = \utilde{\utilde{A}}-\frac{1}{3}tr\utilde{\utilde{A}} $
&% Equation
$ dev\utilde{\utilde{A}} =  \utilde{\utilde{A}}-iso\utilde{\utilde{A}} = \utilde{\utilde{A}}-\frac{1}{3}tr\utilde{\utilde{A}} $
&% Units
$ - $
%&% Derivation
%$ \utilde{\utilde{A}} = iso\utilde{\utilde{A}} + dev\utilde{\utilde{A}} $
&% Notes
-
\\[1ex]
% #20 ----------------------------------------------------------------------
\hline
\\[-1ex]
% Name
Spherical Deformation
&% Purpose
-
&% My Symbol
-
&% Equation
$ \utilde{\utilde{F}}=\alpha\utilde{\utilde{I}} $
&% Equation
$ \utilde{\utilde{F}}=\alpha\utilde{\utilde{I}} $
&% Units
-
%&% Derivation
%
&% Notes
Volume change without shape change
\\[1ex]
% #21 ----------------------------------------------------------------------
\hline
\\[-1ex]
% Name
Isochoric Deformation
&% Purpose
-
&% My Symbol
-
&% Equation
$ J = det\utilde{\utilde{F}} = 1 $
&% Equation
$ J = det\utilde{\utilde{F}} = 1 $
&% Units
-
%&% Derivation
%-
&% Notes
Shape change without volume change
\\[1ex]
% #22 ----------------------------------------------------------------------
\hline
\\[-1ex]
% Name
Hooke's Law
&% Purpose
Relate stress to strain
&% My Symbol
-
&% Equation
$ \utilde{\utilde{\sigma}} = \lambda(tr\utilde{\utilde{\varepsilon}})\utilde{\utilde{I}} + 2G\utilde{\utilde{\varepsilon}} $
&% Equation
$ \utilde{\utilde{\sigma}} = \lambda(tr\utilde{\utilde{\varepsilon}})\utilde{\utilde{I}} + 2G\utilde{\utilde{\varepsilon}} $
&% Units
-
%&% Derivation
%-
&% Notes
$\lambda =$ Lam\'{e} constant, $G =$ Shear modulus, $K = \lambda + \frac{2}{3}G =$ Bulk modulus
\\[1ex]
% #23 ----------------------------------------------------------------------
\hline
\\[-1ex]
% Name
Reynolds Transport
&% Purpose
-
&% My Symbol
-
&% Equation
$ \frac{D}{Dt}\int\limits_{\Omega}f\rho dV=\int\limits_{\Omega}\dot{f}\rho dV $
&% Equation
$ \frac{D}{Dt}\int\limits_{\Omega}f\rho dV=\int\limits_{\Omega}\dot{f}\rho dV $
&% Units
-
%&% Derivation
%See Leibniz Theorem
&% Notes
$ \dot{\phi} = \frac{D\phi}{Dt} = (\frac{\partial \phi}{\partial t})_{\utilde{X}} $ Lagrange rate
\\[1ex]
% #26 ----------------------------------------------------------------------
\hline
\\[-1ex]
% Name
Material Velocity (Lagrangian)
&% Purpose
-
&% My Symbol
$ \utilde{v}(\utilde{X},t) $
&% Equation
$ \utilde{v} =(\frac{\partial \utilde{x}}{\partial t})_{\utilde{X}} $
&% Equation
$ \utilde{v} =(\frac{\partial \utilde{x}}{\partial t})_{\utilde{X}} $
&% Units
$ \frac{Length}{Second} $
%&% Derivation
%-
&% Notes
This is the Lagrangian (particle tracking) velocity $\utilde{v}(\utilde{X},t)$
\\[1ex]
% #27 ----------------------------------------------------------------------
\hline
\\[-1ex]
% Name
Material Velocity (Eulerian)
&% Purpose
-
&% My Symbol
$ \utilde{v}(\utilde{x},t) $
&% Equation
$ \utilde{v} =(\frac{\partial \utilde{x}}{\partial t})_{\utilde{X}}$, Substitute $\utilde{X}(\utilde{x},t)$ for $\utilde{X}$
&% Equation
$ \utilde{v} =(\frac{\partial \utilde{x}}{\partial t})_{\utilde{X}}$, Substitute $\utilde{X}(\utilde{x},t)$ for $\utilde{X}$
&% Units
$ \frac{Length}{Second} $
%&% Derivation
%-
&% Notes
This is Eulerian (stationary observer) velocity $\utilde{v}(\utilde{x},t)$
\\[1ex]
% #28 ----------------------------------------------------------------------
\hline
\\[-1ex]
% Name
Reference Backward Gradient
&% Purpose
Alternate method to find $\dot{\utilde{\utilde{F}}}$
&% My Symbol
$ \utilde{v}\overleftarrow{\nabla_{0}} $
&% Equation
$ \utilde{v}\overleftarrow{\nabla_{0}} = (\frac{\partial \utilde{v}}{\partial \utilde{X}})_{t}=\dot{\utilde{\utilde{F}}} $
&% Equation
$ \utilde{v}\overleftarrow{\nabla_{0}} = (\frac{\partial \utilde{v}}{\partial \utilde{X}})_{t}=\dot{\utilde{\utilde{F}}} $
&% Units
$ \frac{1}{Second} $
%&% Derivation
%
&% Notes
Use the Lagrange material velocity $\utilde{v}(\utilde{X},t)$
\\[1ex]
% #29 ----------------------------------------------------------------------
\hline
\\[-1ex]
% Name
Spatial Backward Gradient
&% Purpose
Alternate method to find the velocity gradient ($\utilde{\utilde{L}}$)
&% My Symbol
$ \utilde{v}\overleftarrow{\nabla} $
&% Equation
$ \utilde{v}\overleftarrow{\nabla} =(\frac{\partial \utilde{v}}{\partial \utilde{x}})_{t}=\utilde{\utilde{L}} $
&% Equation
$ \utilde{v}\overleftarrow{\nabla} =(\frac{\partial \utilde{v}}{\partial \utilde{x}})_{t}=\utilde{\utilde{L}} $
&% Units
$ \frac{1}{Second} $
%&% Derivation
%-
&% Notes
Use the Lagrange material velocity $\utilde{v}(\utilde{X},t)$
\\[1ex]
% #30 ----------------------------------------------------------------------
\hline
\\[-1ex]
% Name
Polar Decomposition
&% Purpose
Decomp $\utilde{\utilde{F}}$ into a `stretch' and `rotate'
&% My Symbol
-
&% Equation
$ \utilde{\utilde{F}}=\utilde{\utilde{R}}\cdot\utilde{\utilde{U}}=\utilde{\utilde{V}}\cdot\utilde{\utilde{R}} $
&% Equation
$ \utilde{\utilde{F}}=\utilde{\utilde{R}}\cdot\utilde{\utilde{U}}=\utilde{\utilde{V}}\cdot\utilde{\utilde{R}} $
&% Units
-
%&% Derivation
%-
&% Notes
 
\\[1ex]
% #31a----------------------------------------------------------------------
\hline
\\[-1ex]
% Name
Right Stretch Tensor
&% Purpose
Used in polar decomposition
&% My Symbol
$ \utilde{\utilde{U}} $
&% Equation
$ \utilde{\utilde{U}} =\utilde{\utilde{C}}^{\frac{1}{2}}$
&% Equation
$ \utilde{\utilde{U}} =\utilde{\utilde{C}}^{\frac{1}{2}}$
&% Units
$ 1 $
%&% Derivation
%-
&% Notes
$\utilde{\utilde{U}}=\utilde{\utilde{R}}^{T}\cdot\utilde{\utilde{F}}$
\\[1ex]
% #31b----------------------------------------------------------------------
\hline
\\[-1ex]
% Name
Left stretch tensor
&% Purpose
Used in polar decomposition
&% My Symbol
$ \utilde{\utilde{V}} $
&% Equation
$ \utilde{\utilde{V}} =\utilde{\utilde{B}}^{\frac{1}{2}}$
&% Equation
$ \utilde{\utilde{V}} =\utilde{\utilde{B}}^{\frac{1}{2}}$
&% Units
$ 1 $
%&% Derivation
%-
&% Notes
$\utilde{\utilde{V}}=\utilde{\utilde{F}}\cdot\utilde{\utilde{R}}^{T}$
\\[1ex]




% #31 ----------------------------------------------------------------------
\hline
\\[-1ex]
% Name

&% Purpose

&% My Symbol
$ \utilde{\utilde{H}} $
&% Equation
$ \utilde{\utilde{H}} = \utilde{u}\overleftarrow{\nabla_{0}}=(\frac{\partial \utilde{u}}{\partial \utilde{X}})_{t}$
&% Equation
$ \utilde{\utilde{H}} = \utilde{u}\overleftarrow{\nabla_{0}}=(\frac{\partial \utilde{u}}{\partial \utilde{X}})_{t}$
&% Units
$ 1 $
%&% Derivation
%-
&% Notes
$\utilde{u}=\utilde{x}-\utilde{X}$, $\utilde{\utilde{E}}=\frac{1}{2}(\utilde{\utilde{H}}+\utilde{\utilde{H}}^{T}+\utilde{\utilde{H}}^{T}\cdot\utilde{\utilde{H}})$
\\[1ex]
% #32 ----------------------------------------------------------------------
\hline
\\[-1ex]
% Name

&% Purpose

&% My Symbol
$ \utilde{\utilde{h}} $
&% Equation
$ \utilde{\utilde{h}} = \utilde{u}\overleftarrow{\nabla}=(\frac{\partial \utilde{u}}{\partial \utilde{x}})_{t}$
&% Equation
$ \utilde{\utilde{h}} = \utilde{u}\overleftarrow{\nabla}=(\frac{\partial \utilde{u}}{\partial \utilde{x}})_{t}$
&% Units
$ 1 $
%&% Derivation
%-
&% Notes
$\utilde{u}=\utilde{x}-\utilde{X}$, $\utilde{\utilde{e}}=\frac{1}{2}(\utilde{\utilde{h}}+\utilde{\utilde{h}}^{T}-\utilde{\utilde{h}}^{T}\cdot\utilde{\utilde{h}})$
\\[1ex]
% #33 ----------------------------------------------------------------------
\hline
\\[-1ex]
% Name
Spatial Gradient
&% Purpose
S.G. of scalar field
&% My Symbol
$ \overrightarrow{\nabla}\phi $
&% Equation
$ \overrightarrow{\nabla}\phi = \frac{\partial \phi}{\partial \utilde{x}} $
&% Equation
$ \overrightarrow{\nabla}\phi = \frac{\partial \phi}{\partial \utilde{x}} $
&% Units
-
%&% Derivation
%-
&% Notes
 
\\[1ex]
% #34 ----------------------------------------------------------------------
\hline
\\[-1ex]
% Name
Spatial Gradient
&% Purpose
S.G. of vector field
&% My Symbol
$ \overrightarrow{\nabla}\utilde{\phi}$,   $\utilde{\phi}\overleftarrow{\nabla}$
&% Equation
$(\overrightarrow{\nabla}\utilde{\phi})_{ij} = \frac{\partial \phi_{j}}{\partial x_{i}}$, $(\utilde{\phi}\overleftarrow{\nabla})_{ij} = \frac{\partial \phi_{i}}{\partial x_{j}}$
&% Equation
$(\overrightarrow{\nabla}\utilde{\phi})_{ij} = \frac{\partial \phi_{j}}{\partial x_{i}}$, $(\utilde{\phi}\overleftarrow{\nabla})_{ij} = \frac{\partial \phi_{i}}{\partial x_{j}}$
&% Units
-
%&% Derivation
%-
&% Notes
$\overrightarrow{\nabla}r = \frac{\utilde{x}}{r}$,   $\overrightarrow{\nabla}\utilde{\phi}=(\utilde{\phi}\overleftarrow{\nabla})^{T}$
\\[1ex]
% #35 ----------------------------------------------------------------------
\hline
\\[-1ex]
% Name
Divergence
&% Purpose
Measures magnitude of outward flux of a vector field
&% My Symbol
$ \overrightarrow{\nabla}\cdot\utilde{\phi} = \utilde{\phi}\overleftarrow{\nabla} $
&% Equation
$ (\overrightarrow{\nabla}\cdot\utilde{\phi}) = (\utilde{\phi}\overleftarrow{\nabla}) = \frac{\partial \phi_{i}}{\partial x_{i}} $
&% Equation
$ (\overrightarrow{\nabla}\cdot\utilde{\phi}) = (\utilde{\phi}\overleftarrow{\nabla}) = \frac{\partial \phi_{i}}{\partial x_{i}} $
&% Units
-
%&% Derivation
%-
&% Notes

\\[1ex]
% #36 ----------------------------------------------------------------------
\hline
\\[-1ex]
% Name
Curl
&% Purpose
Describes the `rotation' of a vector field
&% My Symbol
$ \overrightarrow{\nabla}\times\utilde{\phi} $
&% Equation
$ (\overrightarrow{\nabla}\times\utilde{\phi})_{i} =\varepsilon_{ijk}\frac{\partial \phi_{k}}{\partial x_{j}}$,  $(\utilde{\phi}\times\overleftarrow{\nabla})_{i}=\varepsilon_{ijk}\frac{\partial \phi_{j}}{\partial x_{k}}$
&% Equation
$ (\overrightarrow{\nabla}\times\utilde{\phi})_{i} =\varepsilon_{ijk}\frac{\partial \phi_{k}}{\partial x_{j}}$,  $(\utilde{\phi}\times\overleftarrow{\nabla})_{i}=\varepsilon_{ijk}\frac{\partial \phi_{j}}{\partial x_{k}}$
&% Units
-
%&% Derivation
%-
&% Notes
$ \overrightarrow{\nabla}\times\utilde{\phi} =-(\utilde{\phi}\times\overleftarrow{\nabla}) $
\\[1ex]
% #37 ----------------------------------------------------------------------
\hline
\\[-1ex]
% Name
Direction Cosine Matrix
&% Purpose
Basis transformation
&% My Symbol
$ \utilde{\utilde{L}} $
&% Equation
$ L_{ij}=e_{i}^{A}e_{j}^{B} = \cos\theta_{ij}^{AB} $
&% Equation
$ L_{ij}=e_{i}^{A}e_{j}^{B} = \cos\theta_{ij}^{AB} $
&% Units
$ 1 $
%&% Derivation
%-
&% Notes
$\utilde{\utilde{L}}=\utilde{\utilde{L}}^{T}$
\\[1ex]
% #38 ----------------------------------------------------------------------
\hline
\\[-1ex]
% Name
Cayley-Hamilton Theorem
&% Purpose
-
&% My Symbol
-
&% Equation
$\utilde{\utilde{A}}^{3}-I_{1}\utilde{\utilde{A}}^{2}+I_{2}\utilde{\utilde{A}}-I_{3} \utilde{\utilde{I}}= \utilde{\utilde{0}} $
&% Equation
$A_{ik} A_{kl} A_{lj} - I_1 A_{ik} A_{kj} + I_2 A_{ij} - I_3 \delta_{ij} = 0_{ij}   $
&% Units
-
%&% Derivation
%-
&% Notes
$\lambda^{3}-I_{1}\lambda^{2}+I_{2}\lambda-I_{3}=0$
\\[1ex]
% #39 ----------------------------------------------------------------------
\hline
\\[-1ex]
% Name
Internal Dissipation
&% Purpose
-
&% My Symbol
$ \mathcal{D} $
&% Equation
$ \mathcal{D} = \mathcal{P}_{s} + \theta\dot{\eta}-\dot{e} $
&% Equation
$ \mathcal{D} = \mathcal{P}_{s} + \theta\dot{\eta}-\dot{e} $
&% Units
1
%&% Derivation
%$\mathcal{D}-\frac{1}{\rho\theta}\utilde{q}\cdot\overrightarrow{\nabla}\theta \geq 0 $
&% Notes
$\mathcal{D} \geq 0$ and $\eta$ is entropy per mass
\\[1ex]
% #40 ----------------------------------------------------------------------
\hline
\\[-1ex]
% Name
Isotropic  Tensors
&% Purpose
Second and fourth order
&% My Symbol
-
&% Equation
$A_{ij}=\alpha I_{ij}$,   $A_{ijkl}=\alpha\delta_{ij}\delta_{kl}+\beta\delta_{ik}\delta_{jl}+\gamma\delta_{il}\delta{jk}$
&% Equation
$A_{ij}=\alpha I_{ij}$,   $A_{ijkl}=\alpha\delta_{ij}\delta_{kl}+\beta\delta_{ik}\delta_{jl}+\gamma\delta_{il}\delta{jk}$
&% Units
-
%&% Derivation
%-
&% Notes
if $A_{ijkl}$ is minor symmetric $\beta=\gamma$
\\[1ex]
% #41 ----------------------------------------------------------------------
\hline
\\[-1ex]
% Name
Isotropic Function
&% Purpose
-
&% My Symbol
-
&% Equation
$\phi(\utilde{\utilde{Q}}\cdot\utilde{v}) = \phi(\utilde{v})\hspace{20pt}for all \utilde{\utilde{Q}}=\utilde{\utilde{Q}}^{-T}$
&% Equation
$\phi(\utilde{\utilde{Q}}\cdot\utilde{v}) = \phi(\utilde{v})\hspace{20pt}for all \utilde{\utilde{Q}}=\utilde{\utilde{Q}}^{-T}$
&% Units
-
%&% Derivation
%-
&% Notes
For tensors $\phi(\utilde{\utilde{Q}}\cdot\utilde{\utilde{T}}\cdot\utilde{\utilde{Q}}^{T})$
\\[1ex]
% #42 ----------------------------------------------------------------------
\hline
\\[-1ex]
% Name
Stress Power
&% Purpose
-
&% My Symbol
$ \mathcal{P}_{s} $
&% Equation
$ \mathcal{P}_{s} = \frac{1}{\rho}\utilde{\utilde{\sigma}}:\utilde{\utilde{L}}=\frac{1}{\rho_{0}}\utilde{\utilde{S}}:\dot{\utilde{\utilde{E}}}$
&% Equation
$ \mathcal{P}_{s} = \frac{1}{\rho}\utilde{\utilde{\sigma}}:\utilde{\utilde{L}}=\frac{1}{\rho_{0}}\utilde{\utilde{S}}:\dot{\utilde{\utilde{E}}}$
&% Units
$ \frac{Watt}{kg} $
%&% Derivation
%$\int\limits_{\partial \Omega}\utilde{v}\cdot\utilde{\utilde{\sigma}}\cdot\utilde{n}dS=\int\limits_{\Omega}(\utilde{v}\cdot\utilde{\utilde{\sigma}})\cdot\overleftarrow{\nabla}dV$
&% Notes
1)Product Rule 2)use equation of motion $\utilde{\utilde{\sigma}}\cdot\overleftarrow{\nabla} = \rho\utilde{a}-\rho\utilde{b}$
\\[1ex]
% #43 ----------------------------------------------------------------------
\hline
\\[-1ex]
% Name
Mechanical Power
&% Purpose

&% My Symbol
$ \mathcal{P}_{m} $
&% Equation
$ \mathcal{P}_{m} =\int\limits_{\partial \Omega}\utilde{t}\cdot\utilde{v}dS + \int\limits_{\Omega}\utilde{b}\cdot\utilde{v}\rho dV$
&% Equation
$ \mathcal{P}_{m} =\int\limits_{\partial \Omega}\utilde{t}\cdot\utilde{v}dS + \int\limits_{\Omega}\utilde{b}\cdot\utilde{v}\rho dV$
&% Units
$ \frac{Watt}{kg} $
%&% Derivation
%-
&% Notes
 
\\[1ex]
% #44 ----------------------------------------------------------------------
\hline
\\[-1ex]
% Name
Nanson's Relation
&% Purpose
Tracks area
&% My Symbol
-
&% Equation
$ d\utilde{A} = J \utilde{\utilde{F}}^{-T}\cdot d\utilde{A}_{0} \hspace{15pt} dA\utilde{n}=J \utilde{\utilde{F}}^{-T}\cdot\utilde{N} dA_{0}$
&% Equation
$ d\utilde{A} = J \utilde{\utilde{F}}^{-T}\cdot d\utilde{A}_{0} \hspace{15pt} dA\utilde{n}=J \utilde{\utilde{F}}^{-T}\cdot\utilde{N} dA_{0}$
&% Units
-
%&% Derivation
%$ (\utilde{\utilde{F}}\cdot\utilde{u})\times(\utilde{\utilde{F}}\cdot\utilde{v})=\utilde{\utilde{F}}^{c}\cdot(\utilde{u}\times\utilde{v})$
&% Notes
let $\utilde{u} = d\utilde{X}^{1}$ and $\utilde{v}=d\utilde{X}^{2}$, $(\utilde{u}\times\utilde{v})=d\utilde{A}_{0}$
\\[1ex]
% #44 ----------------------------------------------------------------------
\hline
\\[-1ex]
% Name
Local Equation of Motion
&% Purpose
Continuum analog of $F=ma$
&% My Symbol
-
&% Equation
$ \overrightarrow{\nabla}\cdot\utilde{\utilde{\sigma}}+\rho\utilde{b}=\rho\utilde{a}$
&% Equation
$ \overrightarrow{\nabla}\cdot\utilde{\utilde{\sigma}}+\rho\utilde{b}=\rho\utilde{a}$
&% Units
-
%&% Derivation
%$ \int\limits_{\partial \Omega}\utilde{\utilde{\sigma}}\cdot\utilde{n}dA + \int\limits_{\Omega}\rho\utilde{b}dV=\frac{D}{Dt}\int\limits_{\Omega}\utilde{v}\rho dV$
&% Notes
(surface forces)+(body forces) = (rate of momentum)
\\[1ex]
% #45 ----------------------------------------------------------------------
\hline
\\[-1ex]
% Name
Continuity
&% Purpose
Local form of conservation of mass
&% My Symbol
-
&% Equation
$ \dot{\rho}+\rho\overrightarrow{\nabla}\cdot\utilde{v}=0 $
&% Equation
$ \dot{\rho}+\rho\overrightarrow{\nabla}\cdot\utilde{v}=0 $
&% Units
-
%&% Derivation
%-
&% Notes
alternate form: $\rho_{,t}+\overrightarrow{\nabla}\cdot(\rho\utilde{v})=0$
\\[1ex]
% #46 ----------------------------------------------------------------------
\hline
\\[-1ex]
% Name
First Law of Thermodynamics
&% Purpose
Conservation of energy
&% My Symbol
-
&% Equation
$\dot{e}=\frac{1}{\rho}\utilde{\utilde{\sigma}}:\utilde{\utilde{D}}+\xi-\frac{1}{\rho}\overrightarrow{\nabla}\cdot\utilde{q}$
&% Equation
$\dot{e}=\frac{1}{\rho}\utilde{\utilde{\sigma}}:\utilde{\utilde{D}}+\xi-\frac{1}{\rho}\overrightarrow{\nabla}\cdot\utilde{q}$
&% Units
-
%&% Derivation
%-
&% Notes
$\xi=$`microwave heat', $e=$internal energy per mass, $\utilde{q}=$heat flux
\\[1ex]
% #47 ----------------------------------------------------------------------
\hline
\\[-1ex]
% Name
Second Law of Thermodynamics - Local Form
&% Purpose
Disorder tends to increase. 
&% My Symbol
-
&% Equation
$ \dot{\eta} \geq \frac{\xi}{\theta}-\frac{1}{\rho}\overrightarrow{\nabla}\cdot(\frac{\utilde{q}}{\theta})$
&% Equation
$ \dot{\eta} \geq \frac{\xi}{\theta}-\frac{1}{\rho}\overrightarrow{\nabla}\cdot(\frac{\utilde{q}}{\theta})$
&% Units
-
%&% Derivation
%
&% Notes
A.K.A. Clausius-Duhem Inequality
\\[1ex]
% #01 ----------------------------------------------------------------------
\\[-1ex]
% Name
Traction
&% Purpose
-
&% My Symbol
$ \tu{t} $
&% Equation
$ \tu{t}=\tuu{\sigma}\cdot\tu{n} $
&% Equation
$ \tu{t}=\tuu{\sigma}\cdot\tu{n} $
&% Units
$ \frac{Force}{Area} $
%&% Derivation
%-
&% Notes
Cauchy Tetrahedron Argument proves $\tu{t}$ is linear
\\[1ex]
% #02 ----------------------------------------------------------------------
\hline
\\[-1ex]
% Name
Eulerian Rate
&% Purpose
Rate seen by fixed observer (optical sensor)
&% My Symbol
$ \phi_{,t} $
&% Equation
$ \phi_{,t}=(\frac{\partial \phi}{\partial t})_{\tu{x}} $
&% Equation
$ \phi_{,t}=(\frac{\partial \phi}{\partial t})_{\tu{x}} $
&% Units
$ \frac{1}{Second} $
%&% Derivation
%-
&% Notes
$(\frac{\partial \phi}{\partial t})_{\tu{X}} = (\frac{\partial \phi}{\partial t})_{\tu{x}}+(\frac{\partial \phi}{\partial \tu{x}})_{t}\cdot(\frac{\partial \tu{x}}{\partial t})_{\tu{X}} $
\\[1ex]
% #03 ----------------------------------------------------------------------
\hline
\\[-1ex]
% Name
Lagrange Rate/Material Rate
&% Purpose
Rate as experienced by discrete particles
&% My Symbol
$ \dot{\phi} $
&% Equation
$ \dot{\phi} = \frac{D\phi}{Dt} = (\frac{\partial \phi}{\partial t})_{\tu{X}} $
&% Equation
$ \dot{\phi} = \frac{D\phi}{Dt} = (\frac{\partial \phi}{\partial t})_{\tu{X}} $
&% Units
$ \frac{1}{Second} $
%&% Derivation
%-
&% Notes
Lagrange rates are the `usual' rates
\\[1ex]
% #06 ----------------------------------------------------------------------
% TODO
\hline
\\[-1ex]
% Name
Leibniz Theorem
&% Purpose
Lemma for Reynold's Transport
&% My Symbol
-
&% Equation
$\frac{d}{dt}\int\limits_{\Omega(t)}f(\utilde{x},t)dV = \int\limits_{\Omega(t)}\frac{\partial f(\utilde{x},t)}{\partial t}dV + \int\limits_{\Omega(t)}f(\utilde{x},t)\utilde{v}_{x}\cdot\utilde{n}dS$
&% Equation
$\frac{d}{dt}\int\limits_{\Omega(t)}f(\utilde{x},t)dV = \int\limits_{\Omega(t)}\frac{\partial f(\utilde{x},t)}{\partial t}dV + \int\limits_{\Omega(t)}f(\utilde{x},t)\utilde{v}_{x}\cdot\utilde{n}dS$
&% Units
-
%&% Derivation
%-
&% Notes
Rate of an integral over a time varying domain. $dS=$ surface velocity.
\\[1ex]






% #25 ----------------------------------------------------------------------
\hline
\\[-1ex]
% Name
Material Rate
&% Purpose
-
&% My Symbol
$ \dot{\utilde{\utilde{F}}} $
&% Equation
$ \dot{\utilde{\utilde{F}}} =(\frac{\partial \utilde{\utilde{F}}}{\partial t})_{\utilde{X}} $
&% Equation
$ \dot{\utilde{\utilde{F}}} =(\frac{\partial \utilde{\utilde{F}}}{\partial t})_{\utilde{X}} $
&% Units
$ \frac{1}{Second} $
%&% Derivation
%-
&% Notes
This is the Lagrangian rate (as $\utilde{X}$ is constant)
\\[1ex]




\hline


%\\[-1ex]
%% Name
%
%&% Purpose
%
%&% My Symbol
%$$
%&% Accepted Symbol
%$$
%&% Equation
%$$
%&% Units
%$$
%%&% Derivation
%%$$
%&% Notes
%
%\\[1ex]
\end{tabular}
\\[1ex]
Created by M. Scot Swan, 2017
\end{center}
\end{document}
